\section{Plans for the more distant future}
First, I will update our comparison study with Niels Warburton's geodesic code, to higher l-modes and more recent versions of both codes and report the results to Niels Warburton and Peter Diener.  

I also plan to run Peter Diener's simulation of scalar self-force on a Schwarzschild background for generic orbits including a back-reaction which causes the orbit to evolve away from a geodesic, and examine the data, with the purpose of understanding the physical implications. I will compare Warburton's simulation based on frequency domain initial data and geodesic evolution to Peter Diener's self-consistent time domain code. In the time domain, the state of the field itself naturally accounts for the past orbit of the particle when computing the self force at a given time, assuming sufficient orders of perturbation theory are included. Peter Diener, Ian Vega, Barry Wardell, and Steven Detweiler~\cite{diener_vega_wardell_detweiler_2012} have previously published on self-consistent evolution of a particle around a Schwarzschild black-hole; however, it did not have sufficient accuracy. We are attempting to improve the accuracy using entirely different numerical methods.  in one dimension with spherical harmonics instead of in 1+3 dimensions and using the Discontinuous Galerkin method. 

\section{Self-consistent evolution}
To extend the wave equation to that produced by a particle on a self-consistent orbit, it is necessary to include several additional effects. In addition to the wave equation with a source, the acceleration evolves according to a simplified version of the geodesic equation applied to a scalar particle. The particle also gains or loses mass equal to the work being done on it. 

\begin{eqnarray}
  \Box\Psi^{ret} = -4\pi q \int\delta_4(x,z(\tau^\prime))d\tau^\prime\\
    ma^\alpha=q(g^{\alpha\beta}_{(0)}+u^\alpha u^\beta)\Psi^{R}_{,\beta}\\
    \frac{dm}{d\tau}=-q u^\alpha\Psi^R_{,\alpha}
    \label{genericev}
\end{eqnarray}
The second equation gives the back-reaction due to acceleration of the particle. Here, $\Psi^R$ is the regularized field. The third equation governs the self-consistent evolution of the mass of the particle.~\cite{WardellSelfForceReview}

There are two methods for evolving the orbit that we may use, already implemented in the code by Peter Diener: geodesic evolution and osculating orbits~\cite{pound_poisson}.

\subsection{Geodesic evolution}
The geodesic equation is modified to include a force term on the right hand side in the presence of a self-force or external force~\cite{Carroll}.
\begin{equation}
  \frac{d^2x^\mu}{d\tau^2}+\Gamma^\mu_{\rho\sigma}\frac{dx^\rho}{d\tau}\frac{dx^\sigma}{d\tau}=a^\mu
\end{equation}
This equation, together with Equations~\ref{genericev}, provide the basis for the generic evolution code when the geodesic evolution method is used. 

\subsection{Osculating orbits}

An alternative approach is possible, based upon Reference~\cite{pound_poisson}. In a Schwarzschild spacetime, if the effect of the small black hole is neglected, there is a Killing vector along the time direction and along all three spatial directions, resulting in linear momentum conservation in all directions and hence angular momentum conservation. It is natural to evolve in a physical process that is closely related to these quantities. The eccentric orbit geodesic parameters $p$ and $e$ (semilatus rectum and eccentricity) are chosen (see Chapter~\ref{ellipticalorb} In the self-consistent evolution, the orbit is accelerated from one geodesic to a neighboring geodesic, and gradually evolves through touching geodesics over time. In this process, $p$ and $e$ are updated via a series of ordinary differential equations with extraordinarily complicated right hand sides. This is done using the same RK4 routine that is used to solve the wave equation. It will hopefully be more accurate than the geodesic evolution scheme because angular parameters $\phi$ and $\chi$ monotonically and smoothly increase while $p$ and $e$ evolve slowly, reducing roundoff error. This is in contrast to the accumulation of error introduced through oscillations in $r$ and $t$ in the geodesic method. In the self-consistent approach, the mass and acceleration will also be evolved, eventually. 






