\section{Plans for the more distant future}
First, I will update our comparison study with Niels Warburton's geodesic code, to higher l-modes and more recent versions of both codes and report the results to Niels Warburton and Peter Diener.  

I also plan to run Peter Diener's simulation of scalar self-force on a Schwarzschild background for generic orbits and examine the data, with the purpose of understanding the physicsal implications of the output of the simulation. If necessary, I may help debug the simulation. Peter Diener, Ian Vega, Barry Wardell, and Steven Detweiler~\cite{diener_vega_wardell_detweiler} have previously published on self-consistent evolution of a particle around a Schwarzschild blackhole; however, the current goal is to reimplement the same thing more precisely, in 1 dimension with spherical harmonics instead of in 1+3 dimensions and using the Discontinuous Galerkin method. 

\section{Generic orbits}
To extend the wave equation to that produced by a particle on a self-consistent orbit, it is necessary to include several additional effects.

\begin{eqnarray}
  (\Box - \xi R)\Psi^{ret} = -4\pi q \int\delta_4(x,z(\tau^\prime))d\tau^\prime\\
    ma^\alpha=q(g^{\alpha\beta}_{(0)}+u^\alpha u^\beta)\Psi^{R}_{,\beta}\\
    \frac{dm}{d\tau}=-q u^\alpha\Psi^R_{,\alpha}
    \label{genericev}
\end{eqnarray}
R is the Ricci scalar (0 in Schwarzchild spacetime) and $\xi$ is the coupling to curvature. The first equation gives th scalar wave equation in curved spacetime, with a source. The second equation gives the back-reaction due to acceleration of the particle. Here, $\Psi^R$ is the regularized field. The third equation governs the self-consistent evolution of the mass of the particle.~\cite{WardellSelfForceReview}

There are two methods for evolving the orbit that we may use, already implemented in the code by Peter Diener: geodesic evolution and osculating orbits~\ref{pound_poisson}.

\subsection{Geodesic evolution}
The geodesic equation is modified to include a force term on the right hand side in the presence of a self-force or external force~\cite{carroll}.
\begin{equation}
  \frac{d^2x^\mu}{d\tau^2}+\Gamma^\mu_{\rho\sigma}\frac{dx^\rho}{d\tau}\frac{dx^\sigma}{d\tau}=qF^\mu_\nu\frac{dx^\nu}{d\tau}=a^\mu
\end{equation}
This equation, together with Equations~\ref{genericev}, provide the basis for the generic evolution code when the geodesic evolution method is used. 

\subsection{Osculating orbits}

An alternative approach is possible, based upon Reference~\cite{pound_poisson}. In an evolving orbit, the angular momentum and energy of a particle are only approximate symmetries, but extracted far away, they become measurable quantities. It is natural to evolve in a physical process that is closely related to these quantities. The eccentric orbit geodesic parameters $p$ and $e$ (semilatus rectum and eccentricity) are chosen (see Chapter~\ref{ellipticalorb} In the self-consistent evolution, the orbit is accelerated from one geodesic to a neighboring geodesic, and gradually evolves through touching geodesics over time. In this process, $p$ and $e$ are updated via a series of ordinary differential equations with extraordinarily complicated right hand sides. This is done using the same RK4 routine that is used to solve the wave equation.






