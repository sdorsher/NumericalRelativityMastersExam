\section{Gravitational Waves}

On February 11, 2016, the LIGO Scientific Collaboration announced the
first detection of gravitational waves from a black hole binary
inspirals, occurring on September 14, 2015, with pre-merger masses of
36 $M_\odot$ and 29 $M_\odot$ and a post merger mass of 62 $M_\odot$
at a redshift of $z=0.09$~\cite{GW150914}. Two subsequent detections
followed, on December 26, 2015~\cite{GW151226} and on January 4,
2017~\cite{GW170104}, with masses that are about the same to within an order of magnitude. This detection was well-anticipated. Russel Hulse and Joseph Taylor were awarded the Nobel Prize in 1993 for their 1974 discovery of a binary pulsar. This, and similar pulsars yielded precise timing measurements of orbital period changes that corresponded to rates expected due to gravitational wave emission~\cite{hulsetaylor}. Stellar mass black holes have been known observationally for some time since the discovery of the x-ray source Cygnus X-1 in the 1960's. Active galactic nuclei at the center of galaxies have been identified as supermassive black holes since the 1960's as well. Saggitarius $A^*$, a supermassive black hole at the center of the Milky Way, was originally identified through radio astronomy and is also observable in the X-ray~\cite{sagastarmultiwavelength}, but its black hole nature has been confirmed most precisely through a 16 year study of the orbits of neighboring stars~\cite{sagastarorbits}. 

All known stellar mass black holes from x-ray sources are spinning~\cite{Bambi2017}, the LIGO gravitational wave detections suggested or were consistent with spinning black holes~\cite{GW150914}~\cite{GW151226}~\cite{GW170104}, and theoretical expectations are that angular moment from the black hole formation process should be deposited in the newly formed black hole, leaving it spinning, for the case of stellar mass black holes. For the case of supermassive black holes, there are multiple formation scenarios. One model suggests that they form from mergers of massive mass black holes from the deaths of Population I stars. Another model suggests that they are formed from large scale structure collapsing first along one dimension, then subsequently along the other two, followed by mergers into larger black holes. The third model suggests a mechanism involving initial black hole seeds at scales between these two, formed from collapsing ``clouds'' in the galactic disk. All of these models impart some spin to the initial seed black hole, that may or may not be retained in subsequent mergers~\cite{formationsmbh}. For the purposes of this document, I consider black holes without spin, using the Schwarzchild metric.


\section{Extreme Mass Ratio Inspirals}
In an extreme mass ratio inspiral (EMRI), a stellar mass black hole, or other compact object such as a white dwarf or neutron star, orbits around a supermassive black hole. It emits energy due to gravitational waves and its orbital frequency gradually increases while its orbital radius gradually decreases. Eventually it crosses the innermost stable circular orbit and, assuming it is on a roughly circular orbit, it plunges into the black hole without completing any further orbits. Picture this in the presence of a negligible accretion disk, or inside the inner edge of the accretion disk, or out of the orbital plane of the accretion disk. 

The stellar mass black hole can be a factor of $10^5$ to $10^6$ smaller than the mass of the supermassive black hole in an EMRI. In an intermediate mass ratio inspiral (IMRI), the mass ratio is more like $10^2$ to $10^4$. LISA will be sensitive to EMRI's and some IMRI's. LIGO may be sensitive to some IRMI's. It is our goal to enable to the LISA community to provide gravitational wave templates to the LISA searches in time for the LISA launch in the early 2030's. Extraordinariy precision is required to perform these searches, requiring an error in the period of no more $\delta P/P<10^{-6}$~\cite{LISA02062017}. This is especially challenging given that the EMRI system must be evolved over a much longer timescale than a single period due to the persistence of the initial conditions in the system and will be much longer than a single period for any physical system. This timescale scales as $M/\mu$, where $M$ is the central black hole mass and $\mu$ is the mass ratio. For a mass ratio of $10^5$, approximately $10^5$ wavecycles must be evolved~\cite{LISA02062017}.

\section{Self force}

Consider a classical model of the atom. This is more like the Bohr model than the plum pudding model, but it doesn't make any attempt at quantization. A negatively charged electron is orbiting a positively charged nucleus. That electron emits radiation and loses energy, becoming more tightly bound. The force that causes it to move inward is called the self-force, because it is the force of the particle interacting with its own electric field ~\cite{dirac1938}.

The self-force also exists in general relativity. In the simplest possible picture of general relativity, a test particle experiencing no force moves on a geodesic. However, once we bring into our picture the the principle that the small particle (now a stellar mass black hole) has mass-- any amount of mass-- it also curves spacetime, generating a back-reaction. It is possible to obtain the self-force in general relativity from either perturbations of the metric~\cite{pound2ndOrderSelfForce2} or perturbations of the particle's worldline~\cite{WardellSelfForceReview}. Either way, the small parameter governing the expansion is the mass ratio of the system. Again, the self-force is due to the interaction of the small black hole with the gravitational field itself.

In this document, we consider a slightly simpler problem, that of a scalar field on a curved spacetime background. In this case, the self-force is still due to the interaction of the scalar field with the particle itself. The stellar mass black hole is now a scalar delta function source that evolves along an orbit through spacetime. As in the other two cases, the source causes scalar waves to be emitted that travel in all spatial directions away from the source, both toward light-like infinity and into the supermassive black hole horizon. The backreaction causes a self-force that propels the particle slightly inward with each orbit as it loses energy to these outgoing waves. Ultimately, if evolved for that long, the particle should enter the horizon of the large black hole and then, it is beyond the scope of present day physics to say what should happen to it. 



\section{The gravitational wave sky}
Currently, there are four distinct windows on the gravitational wave universe planned or in progress. The Laser Interferometer Gravitational Wave Observatory, LIGO, probably deserves first listing, due to their recent success. LIGO observes gravitational waves using a ground based Michelson-Morley interferometer with two 4 kilometer long Fabry-Perot cavity arms. It detects strains as small as $10^{-23} Hz^{-1/2}$~\cite{LIGOsensitivity}. It has thusfar had three detections of blackhole binary pairs in the tens of solar mass range. These mergers have been tracked through inspiral and merger~\cite{GW150914}~\cite{GW151226}~\cite{GW170104}, and ringdown~\cite{LIGO1e}, to produce exact tests of general relativity. LIGO has even been able to set limits on non-General Relativistic behavior, such as the magnitude of the dispersion of the intergalactic medium to gravitational waves~\cite{GW170104} and deviation of the post Newtonian parameters from those predicted from GR~\cite{LIGO1e}. The LIGO-Virgo collaboration continues searches for neutron star binary inspirals, which may have electromagnetic counterparts in the form of supernovae; pulsars, which are thought to have raised ``mountains'' their surface causing a changing quadrupole moment-- these will also have electromagnetic counterparts; gravitational wave bursts correlated with EM, neutrino, or particle sources such as a gamma ray burst; or the cosmic or galactic white dwarf binary stochastic background radiation. In the future, KAGRA, an observatory in Japan, and IndIGO, an observatory in India, will join the search for gravitational waves, refining the collaboration's ability to locate a source on the sky. Geo600 is an existing English gravitational wave detector in England. The Einstein Telescope is a next generation gravitational wave detector that has also been proposed. 


The cosmic microwave background (CMB) can also be used to constrain the primordial stochastic baground of graviational waves. It is a very low frequency source, with wavelengths related to the size of the observable universe ($f\sim 10^{-17} Hz$). Searches depend upon separating the polarization into two components: the B-modes, which have curl, and the E-modes, which don't. If the B-modes and the E-modes are roughly even, then tensor components of the density perturbation dominate, and gravitational waves may be present. Microlensing and dust are significant noise sources that must be carefully subtracted. Current searches include BICEP2, Planck, and the Keck Array. Planned ground and balloon borne expreiments include the ACTPol, Polarbear, CLASS, Piper, and Spider. Planned space-base experiments include COrE, PRISM, LiteBIRD, and PIXIE~\cite{bmodes}. 


Pulsar timing uses the extremely precise timing of pulsars, taking their spin down into account and correlating with pairs of other pulsars. In this timing, they are looking for residual signals that may be due to supermassive black hole binaries, cosmic string cusps, and the cosmic stochastic gravitational wave background. Existing pulsar timing arrays include the Parkes Pulsar Timing Array (PPTA), the Eurpeoan Pulsar Timing Array (EPTA), and the North American Nanohertz Observatory for Gravitational Waves (NANOGrav). They have formed a collaboration called the Interanational Pulsar Timing Array (IPTA). The IPTA is sensitive from $10^{-9}$ to $5\times10^{-8}$ Hz with strains on the order of $10^{-14}$ to $10^{-15}$~\cite{hobbs_dai}.



Extreme mass ratio black hole binaries are likely to be detected only by LISA. The Laser Interferometer Space Observatory (LISA) is a space-based observatory.Like LIGO, it will use interferometry to detect gravitational waves; however, it will have three arms arranged in a triangle separated by 2.5 million kilometers. Each end mirror will be in free-fall on a satellite, with the satellite stabalized by microthrusters. LISA Pathfinder was launched in 2015 and demonstrated the optics essential to this space-based mission can operate successfully remotely. As of June of 2017, LISA is fully funded by the European Space Agency, with a NASA as a lesser partner. The launch date is set in the early 2030's. In addition to EMRI's, LISA may detect IMRI's, supermassive black hole binaries, and the primordial stochastic gravitational wave background.


\section{Notation}
In this manuscript, I use Einstein summation notation for tensors, where a repeated Greek index implies a summation over that repeated index. For example, an $n$ dimensional tensor field of rank (1,2) transforms, in general, according to the rule
\begin{equation}
  T^\alpha_{\beta\gamma}(\bar{x}^1,\ldots,\bar{x}^n)=\Lambda^\alpha_\delta\Lambda^\epsilon_\beta\Lambda^\zeta_\gamma T^\delta_{\epsilon\zeta}(x^1,\ldots,x^n)
\end{equation}
where $\Lambda$ is the jacobian of the coordinate transformation from $x$ to $\bar{x}$. 

Indices are raised by use of the inverse metric and lowered by use of the metric. The metric transforms contravariant one-forms, which constitute the basis, to covariant vectors, which constitute the coordinates, e.g. $u^\beta=g^{\alpha\beta}u_\beta$, where $g^{\alpha\beta}$ is the metric. However, the metric and its inverse can also be used to raise and lower indices of tensors of higher and mixed rank. The metric describes the relative distance between two coordinates on a manifold, in all $n$ dimensions, in an $n\times n$ matrix. Two sign conventions are allowed, depending on whether the time component is positive or negative, though the metric always has a negative determinant in four dimensions. In our sign convention, the Minkowski metric for flat spacetime is given by
\[
\eta^{\mu\nu}=
\begin{bmatrix}
  -1 & 0 & 0 & 0\\
  0 & 1 & 0 & 0\\
  0 & 0 & 1 & 0\\
  0 & 0 & 0 & 1\\
\end{bmatrix}
\]
Here the four dimensions are Cartesian, $t$, $x$, $y$, and $z$. The Schwarzchild metric for a spherically symmetric blackhole without charge or spin is given by
\[
d\tau^2=g^{\mu\nu}
\begin{bmatrix}
  -(1-\frac{2M}{r}) & 0 & 0 & 0\\
  0 & (1-\frac{2M}{r})^{-1} & 0 &0\\
  0 & 0 & r^2 & 0\\
  0 & 0 & 0 & r^2\sin^2\theta
\end{bmatrix}
\]
where $d\tau$ is the proper time, and coordinates are $t$ (the local time), $r$ (a radial coordinate that goes to zero at the singularity, $2M$ at the horizon, and infinity at spatial infinity), $\theta$ (the polar angle), and $\phi$ (the azimuthal angle). To obtain the inverse (lowered) metric, simply invert the matrix representation.


